\documentclass[a4paper]{scrartcl}

\usepackage[utf8]{inputenc}
\usepackage[ngerman]{babel}		% ö,ä,ü und silbentrennung
\hyphenpenalty=100000			% keine Silbentrennung => bessere Lesbarkeit
\usepackage[onehalfspacing]{setspace}
\usepackage{arydshln}

\usepackage{amsmath, amsfonts, amssymb, gensymb, mathtools}


\renewcommand{\labelenumi}{\alph{enumi})}


\usepackage[margin=3cm]{geometry}
%==============================================================================
%								bold mono-font
%==============================================================================
\usepackage{bold-extra}
%==============================================================================
%								Colors
%==============================================================================
\usepackage{xcolor}
	\definecolor{pblue}{HTML}{4B89DC}		% XML
	\definecolor{pgreen}{HTML}{8CC051}		% Attribute
	\definecolor{pred}{HTML}{DB4455}		% Warning
	\definecolor{datei}{HTML}{F4A460}		%  Dateien

\newcommand{\XML}[1]{\texttt{\textbf{\color{pblue} #1}}}
\newcommand{\Attr}[1]{\texttt{\textbf{\color{pgreen} #1}}}
\newcommand{\Datei}[1]{\texttt{\textbf{\color{datei} #1}}}
%==============================================================================
%								Icons
%==============================================================================
\usepackage[fixed]{fontawesome5}
\newcommand{\icon}[1]{
	\hspace{-2mm}
	\faIcon[solid]{#1}
	\hspace{-2mm}
}
\newcommand{\warning}{{\color{pred}\icon{exclamation-triangle}}}

\title{Anleitung für die Vorbereitung der Map}
\date{}

\begin{document}
	\maketitle
	\vspace{-10mm}
	\section*{Schritt 1: benötigte Datei}
		Erstelle eine \Datei{missionVehicles.xml}. Eine leere Vorlage finden sie im Ordner \Datei{sdk}. Diese Vorlage enthält die Grundstruktur, die für die Kompatibilität mit dieser Mod vorhanden sein müssen. Sollten diese nicht vorhanden sein, dann wird die Backup-Datei der Mod geladen.\\
		Hier nochmal eine Übersicht über die XML-Komponenten:\\
		\begin{tabular}{c|c|p{7cm}}
			\textbf{XML-Komponente} & \textbf{Attribute} & \textbf{Beschreibung}\\
			\hline
			\XML{variants} & & Dies ist der übergreifende Tag, der für die Kompatibilität mit der Mod nötig ist. Alles innerhalb dieses Tags wird von der Mod geladen.\\
			\hline
			\XML{mission} & \Attr{type} & Beschreibt den Missionstyp, für den die \XML{variants} definiert werden. Aktuell werden nur die Missionstypen \texttt{harvest} und \texttt{sow} unterstützt.\\
			\hline
			\XML{variant} & \Attr{name} & Hier wird der Name einer neuen \XML{variant} definiert.\\
			\hdashline
			& \Attr{fruitTypes} & Hier werden die Fruchtarten definiert, für die diese \XML{variant} verwendet werden soll.\\
			\hdashline
			& \Attr{fruitTypeCategories} & Hier kann man die Fruchtkategorien angeben, für deren Früchte die \XML{variant} verwendet werden soll.
			Die \Attr{fruitTypeCategory} \text{'EARTHFRUITS'} wird von der Mod registriert und enthält alle Erdfrüchte.\newline
			Diese werden automatisch erkannt.\\
		\end{tabular}\\
		\newpage
		Ab hier wird die XML vom Standardspiel geladen:\\
		\begin{tabular}{c|c|p{9cm}}
			\textbf{XML-Komponente} & \textbf{Attribute} & \textbf{Beschreibung}\\
			\hline
			\XML{mission} & \Attr{type} &
			Hier wird der Missionstyp angegeben, für den die Fahrzeuge sind. Die Werte von \Attr{type} sind die Missionstypen aus dem Basisspiel.\newline
			Werte: \texttt{harvest}, \texttt{plow}, \texttt{sow}, \texttt{weed}, \texttt{fertilize}, \texttt{spray}, \texttt{cultivate}, \texttt{mow\_bale}\\
			\hline
			\XML{group} & & Dies ist eine Fahrzeuggruppe, die bei einer Mission als Missionsfahrzeuge angeboten werden kann.\\
			\hdashline
			& \Attr{fieldSize} & Definiert ob die Fahrzeuge für ein kleines, mittleres oder großes Feld angeboten werden sollen.\newline
			Werte: \texttt{small}, \texttt{medium}, \texttt{large}\\
			\hdashline
			& \Attr{variant} & Dieser Wert bestimmt, für welche Fruchtarten diese Gruppe ausgewählt werden kann. Hierzu trägt man den Namen der vorher registrierten \XML{variant} ein.\newline
			\warning Damit die missionVehicles.xml auch mit dem Basisspiel ohne diesen Mod problemlos kompatibel ist sollten auf jeden Fall die \XML{variant} \texttt{GRAIN}, \texttt{MAIZE}, \texttt{SUGARBEET}, \texttt{POTATO}, \texttt{COTTON} und \texttt{SUGARCANE} für Ernte- \& Sä-Missionen vorhanden sein.\newline
			\warning Für Mäh-Missionen müssen die \XML{variant} \texttt{HAY} und \texttt{SILAGE} vorhanden sein.\\
			\hline
			\XML{vehicle} & \Attr{filename} & Hier wird der Pfad zu einem einzelnen Missionsfahrzeug spezifiziert. Dieses  Missionsfahrzeug muss als \texttt{storeItem} registriert sein.
		\end{tabular}

	\section*{Schritt 2: Verlinkung der XML-Datei}
		Öffnen Sie die \Datei{map.xml} ihrer Karte und suchen Sie dort den Eintrag \XML{missionVehicles}. Sollte dieser nicht vorhanden sein, fügen Sie die folgende Zeile in ihre \Datei{map.xml} ein:\\
		\XML{<missionVehicles } \Attr{filename}\texttt{="}\texttt{"} \XML{/>}\\
		Tragen Sie nun bei \Attr{filename} den Pfad zu ihrer eigenen \Datei{missionVehicles.xml} beginnend vom Speicherort ihrer \Datei{modDesc.xml} aus. Speichern Sie anschließend die Änderungen.
	\newpage
	\section*{Schritt 3: Selektiertes Aktivieren von Früchten (Optional)}
		\warning Dieser Schritt kann Probleme bei den Missionsfahrzeugen verursachen wenn diese Mod nicht verwendet wird.\\
		\\
		Wenn Sie möchten, können Sie für manche Früchte die Missionen bereits aktivieren. Dies erreichen Sie indem Sie die \Datei{fruitTypes.xml} ihrer Karte öffnen und bei allen Früchten, bei denen die Missionen aktiv sein sollen, das Attribut \Attr{useForFieldJob} auf den Wert \texttt{true} setzen. Umgekehrt gilt: Um die Missionen für diese Früchte zu deaktivieren, setzen die den Wert auf \texttt{false}.\\
		Ob Sie die Missionen aktivieren oder nicht bestimmt auch ob die NPC-Bauern diese Früchte auf ihren Feldern anbauen oder nicht.
\end{document}