\documentclass[a4paper]{scrartcl}

\usepackage[utf8]{inputenc}
\usepackage[ngerman]{babel}		% ö,ä,ü und silbentrennung
\hyphenpenalty=100000			% keine Silbentrennung => bessere Lesbarkeit
\usepackage[onehalfspacing]{setspace}
\usepackage{arydshln}

\usepackage{amsmath, amsfonts, amssymb, gensymb, mathtools}


\renewcommand{\labelenumi}{\alph{enumi})}


\usepackage[margin=3cm]{geometry}
%==============================================================================
%								bold mono-font
%==============================================================================
\usepackage{bold-extra}
%==============================================================================
%								Colors
%==============================================================================
\usepackage{xcolor}
\definecolor{pblue}{HTML}{4B89DC}		% XML
\definecolor{pgreen}{HTML}{8CC051}		% Attribute
\definecolor{pred}{HTML}{DB4455}		% Warning
\definecolor{datei}{HTML}{F4A460}		%  Dateien

\newcommand{\XML}[1]{\texttt{\textbf{\color{pblue} #1}}}
\newcommand{\Attr}[1]{\texttt{\textbf{\color{pgreen} #1}}}
\newcommand{\Datei}[1]{\texttt{\textbf{\color{datei} #1}}}
%==============================================================================
%								Icons
%==============================================================================
\usepackage[fixed]{fontawesome5}
\newcommand{\icon}[1]{
	\hspace{-2mm}
	\faIcon[solid]{#1}
	\hspace{-2mm}
}
\newcommand{\warning}{{\color{pred}\icon{exclamation-triangle}}}

\title{Instructions for preparing a map}
\date{}

\begin{document}
	\maketitle

	\section*{1. Step}
	Create a \Datei{missionVehicles.xml}. You can find an empty template in the folder \Datei{sdk}. This template has the basic structure of a \Datei{missionVehicles.xml} and the needed XML components for the compatibility with this mod.\\
	Here is a little explanation of the components of the XML file:\\
	\begin{tabular}{c|c|p{7cm}}
		\textbf{XML component} & \textbf{Attributes} & \textbf{Description}\\
		\hline
		\XML{variants} & & This is the wrapping component that is needed for the compatibility with this mod. Everything inside this tag is loaded by the mod.\\
		\hline
		\XML{mission} & \Attr{type} & Describes the mission type for that you define the variants. At the moment only the mission types \texttt{harvest} and \texttt{sow} are supported.\\
		\hline
		\texttt{variant} & \texttt{name} & Defines the name of a new variant.\\
		\hdashline
		& \Attr{fruitTypes} & Defines the fruit types for that this variant is used.\\
		\hdashline
		& \Attr{fruitTypeCategories} & Defines the fruit type categories for which the variant should be used.
		The category 'EARTHFRUITS' is registered by the mod and contains all ground fruits like potatos and sugar beet. Those fruits are automatically recognized.
	\end{tabular}\\
	\hspace{0.2cm}\\
	Everything after this point will be loaded by the vanilla game.\\
	\begin{tabular}{c|c|p{9cm}}
		\textbf{XML-Component} & \textbf{Attributes} & \textbf{Description}\\
		\hline
		\XML{mission} & \Attr{type} & Define the mission type for which the groups should be used.\\
		& & The value of \Attr{type} are the mission types. Those are: 'harvest', 'plow', 'sow', 'weed', 'fertilize', 'spray', 'cultivate', 'mow\_bale'\\
		\hline
		\XML{group} & & Defines a group of vehicles. One group will be offered as mission vehicles.\\
		\hdashline
		& \Attr{fieldSize} & Defines if the group should be used for small, medium or large fields.\\
		& & Values: 'small', 'medium', large'\\
		\hdashline
		& \Attr{variant} &
			Defines for which fruit types the group can be selected.
			To ensure that the \Datei{missionVehicles.xml} is compatible with the base game and completely works without this mod the following \XML{variant}s have to be available:\\
		& & For harvesting and sow missions 'GRAIN', 'MAIZE', 'SUGARBEET', 'POTATO', 'COTTON', 'SUGARCANE'\\
		& & For mow and bale missions: 'HAY' and 'SILAGE'\\
		\hline
		\XML{vehicle} & \Attr{filename} & Defines the path to one mission vehicle. This vehicle must be registered as a store item.\\
		\hdashline
		& \Attr{modName} & Defines the name of the mod where the vehicle comes from. For base game vehicle this attribute can be ignored. For modded vehicles this attribute has to be set.
	\end{tabular}

	\section*{2. Step}
	Open the \Datei{map.xml} of your map and search for the entry \XML{missionVehicles}. If this entry isn't available then add the following line into your \Datei{map.xml}:\\
	\XML{<missionVehicles } \Attr{filename}\texttt{="}\texttt{"} \XML{/>}\\
	Now enter at \Attr{filename} the path to your own \Datei{missionVehicles.xml} starting at the root folder of your map. This is the location of your \Datei{modDesc.xml}. And don't forget to save your progress.

	\section*{3. Step (optional)}
	\warning This step could cause problems with mission vehicles if this mod isn't used.\\
	If you want you can enable missions for fruit types. This can be achieved by opening the \Datei{fruitTypes.xml} of your map and for each fruit type where you want to activate field missions set the attribute \Attr{useForFieldJob} to \texttt{true}. Or in reverse: To deactivate missions set the value to \texttt{false}.\\
	If you enable or disable missions for a fruit type this also decides if the AI-farmers plant this fruit type on their fields or not.
\end{document}